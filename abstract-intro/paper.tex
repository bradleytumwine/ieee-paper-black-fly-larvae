\documentclass[conference]{IEEEtran}
%\IEEEoverridecommandlockouts
% The preceding line is only needed to identify funding in the first footnote. If that is unneeded, please comment it out.
\usepackage{cite}
\usepackage{amsmath,amssymb,amsfonts}
\usepackage{algorithmic}
\usepackage{graphicx}
\usepackage{textcomp}
\usepackage{xcolor}
\def\BibTeX{{\rm B\kern-.05em{\sc i\kern-.025em b}\kern-.08em
    T\kern-.1667em\lower.7ex\hbox{E}\kern-.125emX}}


\begin{document}

\title{Low cost IOT based cloud solution for Intelligence Integrated Black fly rearing.}

\author{
    \IEEEauthorblockN{1\textsuperscript{st} Bradley Alfred Mpeera Tuwmine}
    \IEEEauthorblockA{\textit{School of Engineering,CEDAT} \\
        \textit{Makerere University}\\
        Kampala, Uganda \\
        bradleytumwine82@gmail.com}

    \and
    \IEEEauthorblockN{2\textsuperscript{nd} Frank Ssemakula}
    \IEEEauthorblockA{\textit{School of Engineering,CEDAT} \\
        \textit{Makerere University}\\
        Kampala, Uganda \\
        dashdash@gmail.com}

    \and
    \IEEEauthorblockN{3\textsuperscript{rd} Mary Patience Namugwanya}
    \IEEEauthorblockA{\textit{School of Engineering,CEDAT} \\
        \textit{Makerere University}\\
        Kampala, Uganda \\
        nmarypatience@gmail.com}

    \and
    \IEEEauthorblockN{4\textsuperscript{th} Shawal Mbalire}
    \IEEEauthorblockA{\textit{School of Engineering,CEDAT} \\
        \textit{Makerere University}\\
        Kampala, Uganda \\
        mshawal49@icloud.com}\hspace{\linewidth}


}

\maketitle

\begin{abstract}
    This paper aims to increase the production of black fly larvae for farmers to scale up by automating the monotonous and monitoring intesive tasks.
\end{abstract}

\begin{IEEEkeywords}
    black fly soldier fly larvea,iot
\end{IEEEkeywords}
\section{Introduction}

Bio-waste management is one of the main global challenges of our time, with significant
repercussions on human and environmental health related to sanitary issues, pollution of ground
water, and emission of greenhouse gases (GHGs). As global population and consumption rise, biowaste production is also projected to increase significantly. Conventional methods of dealing with
bio-waste, including open dumping in less economically developed countries or landfilling not
equipped with means to capture GHGs such as methane, exacerbate the global warming crisis.
Besides, landfills release various odors, attract disease vectors, and produce leachates that pollute
ground water.

Black soldier fly (BSF), Hermetia illucens, is an insect that has gained popularity among other
insect-based bio-waste treatments for its effectiveness to convert bio-waste using its larvae. Biowaste biomass is consumed by the larvae, which then will be transformed into protein and fat of
the larvae as well as its residue. The larvae is known to be having a high quality protein suitable
for feeding chicken and fish, residue contains nutrients and organic matter that helps reduce soil
depletion. Additionally, its effectiveness in reducing the risk of bacterial transmission through biowaste makes BSF a safe option for bio-waste treatment in farm level.

Previously, BSF farming required wide spaces about 150 m2
that made it difficult to rear in
locations with limited spaces like urban areas. Moreover, the rearing method needs special care to
adjust suitable condition (temperature and humidity) and frequent attention from laborers which
generally take times in each procedure. For this reason, there was a need to establish an intelligent
BSF rearing system.

The Internet of Things (IoT) has begun to emerge as it has helped human in controlling and
monitoring essential conditions using devices which are able to capture, evaluate and transmit
information from the environment to the cloud, where the data will be stored. Environment
monitoring system is one of the most important IoT systems which mainly includes data
collections through sensors and data reviewing for short-term measure as well as remote
management and observations.

By using the IoT system to maintain the environmental condition i.e. temperature, relative
humidity, light intensity and aeration of BSF larvae on its optimum level, data collection from the
sensors needs to be obtained in real time to make sure that the data is accurate and representing
the exact information of the environmental condition, as well as to be saved automatically to the
database for certain period. This is to ensure that the user will have the proper information to
interpret the insect’s environmental condition. Thus, data communication from the sensor to the
database plays an important role in obtaining this need.
This paper seeks to develop an automated solution to BSF rearing addressing two needs:
\begin{enumerate}
    \item Controlling the environmental conditions, which is currently manual and labor-intensive.
    \item Remote monitoring of these conditions in real time
\end{enumerate}

\section{RELATED WORK}

\subsection{Nkozi University.}
In 2021, Joseph Kasana implemented a World Bank funded BSF project in Nkozi university. The
model has the eggs, larvae, pupa and adults in one setup. The setup (enclosure) consists of wooden
cages covered with wire mesh. These he called the love cages. Each cage has a soaked sponge that
provides the BSF with water and contributes towards humidity. The base of a love cage is open
and underneath it is a drum. In this drum, pieces of wood planks are provided in which the adults
lay eggs. The interior of the drum is always dark. The eggs are scraped off the planks and placed
in containers containing feed (wet maize bran). They are thereafter taken to a different building
called a larvaerium where they stay for four days until they become larvae. The larvae are dried
and milled into a powder. The pupa are physically brought into the love cage very early in the
morning (around 4:00 am) when temperatures are still low. This is to prevent any pupa that have
become adults from becoming active and flying away. The entire setup is covered with a blue
plastic. Joseph said this plastic is used to control both temperature and light intensity. Additionally,
it was his belief that the Black Soldier Fly operates best under blue light. Joseph incurred a lot of
losses with this set up and closed the project in November, 2022.

\subsection{Marula Proteen Limited}
Marula Proteen Limited practices BSF rearing on a large scale. They feed urban organic-waste to
BSF larvae. After a short rearing period these larvae are harvested, dried and processed into high
quality protein food for livestock production \cite{b1}. The pupa are taken to a greenhouse at another
location (Namanve). According to the on-site agricultural engineer, the reason for the change in
location is because there are some pests that are harmful to the eggs but liked by the pupa. The
pupa are placed in cages in the greenhouse and provided with ideal conditions to encourage
breeding. The greenhouse is fitted with sensors and a control panel to automatically control
temperature, humidity and light intensity. A fog pump goes on when the humidity is too low, high
energy lamps turn on when the temperatures are low and a motor controls the opening/closing of
an aluminium foil to control light intensity. The eggs laid in-between planks in the cages are
scraped off and taken to an incubator room where they are kept for four days until they develop
into larvae. The incubator is also fitted with sensors to control temperature, light and aeration.

\section{SOFTWARE AND COMPONENTS}
\subsection{Node MCU esp 32}
NodeMCU is an open-source LUA based firmware developed for the ESP8266 wifi chip. By
exploring functionality with the ESP8266 chip, NodeMCU firmware comes with the ESP8266
Development board/kit i.e. NodeMCU Development board. Since NodeMCU is an opensource platform, its hardware design is open for edit/modify/build. NodeMCU Dev Kit/board
consist of ESP8266 wifi enabled chip. The ESP8266 is a low-cost Wi-Fi chip developed by
Espressif Systems with TCP/IP protocol. Additionally, it contains the crucial elements of a
computer: CPU, RAM, networking (WiFi), and even a modern operating system and SDK.
That makes it an excellent choice for Internet of Things (IoT) projects of all kinds. It collects
and transfers data from the sensors to the cloud based server [1] [2].
\subsection{BH1750}
The BH1750 is a calibrated digital light sensor IC that measures the incident light intensity and
converts it into a 16-bit digital number. The sensor directly gives a digital output. The sensor
output can be accessed through an I2C interface. It measures ambient light intensity and the
measurement unit is Lux [3].
\subsection{DHT22}
The DHT-22 (also named as AM2302) is a digital-output relative humidity and temperature
sensor. It uses a capacitive humidity sensor and a thermistor to measure the humidity and
temperature of the surrounding air, and spits out a digital signal on the data pin [4].
\subsection{Relay}
A relay is a programmable electrical switch that can be controlled by a microcontroller. It is
used to programmatically control the switching on/off of devices which use high voltage and/or
high current. For this reason a relay can be thought of as a bridge between the microcontroller
and high voltage devices [5].
\subsection{Servo motor}
Servo motors are great devices that can turn to a specified position. Usually, they have a servo
arm that can turn 180 degrees. Using a microcontroller, we can tell a servo to go to a specified
position and it will go there. As simple as that! Servo motors were first used in the Remote
Control (RC) world, usually to control the steering of RC cars or the flaps on a RC plane. With
time, they found their uses in robotics, automation, and of course, the Arduino world [5]. The
servo motors will act as actuators when a certain condition is met

\subsection{Firebase Realtime Database}
The Firebase Realtime Database is a cloud-hosted database that lets you build rich, collaborative
applications by allowing secure access to the database directly from client-side code. Data is
persisted locally, and even while offline, realtime events continue to fire, giving the end user a
responsive experience. When the device regains connection, the Realtime Database synchronizes
the local data changes with the remote updates that occurred while the client was offline, merging
any conflicts automatically. It provides a flexible, expression-based rules language, called Firebase
Realtime Database Security Rules, to define how your data should be structured and when data
can be read from or written to. When integrated with Firebase Authentication, developers can
define who has access to what data, and how they can access it. It is a NoSQL database and as
such has different optimizations and functionality compared to a relational database. The Realtime
Database API is designed to only allow operations that can be executed quickly. This enables you
to build a great realtime experience that can serve millions of users without compromising on
responsiveness. Because of this, it is important to think about how users need to access your data
and then structure it accordingly. Data is stored as JSON and synchronized in realtime to every
connected client. When you build cross-platform apps with Apple platforms, Android, and
JavaScript SDKs, all of clients share one Realtime Database instance and automatically receive
updates with the newest data [1]
\section{PROPOSED SYSTEM}
\section{ENUMERATION}
Efficient and reliable black soldier fly larvae bio-waste treatment requires a reliable estimate
on the number of larval offspring dosed per unit of bio-waste and/or larvero (treatment crate),
which is determined by laborious manual counting of young larvae \cite{e1}. Additionally, there is
a tendency of BSF farmers to favor larger flies as they have a higher reproduction rate, the
larvae are bigger and thus contain more protein than smaller larvae. Manual selection of such
traits at an industrial level is currently infeasible due to the high numbers of larvae making it
economically unviable \cite{e2}.
Through the deployment of computer vision techniques together with deep learning methods;
1. Larvae enumeration can be made more efficient and accurate.
2. The process can be automated

\section{CONCLUSION}

\section*{Acknowledgment}

We thank Phillip the Chief Technical Officer at Proteen for assistance with hardware like
node MCU ESP32 for prototyping and Joseph Kasana for showing us the project at Nkozi University.

\begin{thebibliography}{00}
    \bibitem{b1}  "Proteen," [Online]. Available: https://weareproteen.com/. [Accessed 23 February 2023].
    \bibitem{e2}  "Towards automatic enumeration of Black Soldier Fly Larvae offspring," [Online]. Available: https://www.eawag.com. [Accessed 23 February 2023].
    \bibitem{e2}  M.F.Hansen, "Towards Machine Vision for Insect Welfare Monitoring and Behavioural Insights," Frontiers in Veterinary Science, vol. 9, 2022.
\end{thebibliography}
\end{document}
